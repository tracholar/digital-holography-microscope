\documentclass[11pt,a4paper]{article}

\linespread{1.3}
\usepackage{geometry}
\geometry{top=1in, bottom=1in,left=1in,right=1in}

\usepackage{fancyhdr}
\pagestyle{fancy}
\lhead{}
\chead{无透镜傅里叶全息显微}
\rhead{左元 SA13006060}
\lfoot{}
\cfoot{\thepage}
\rfoot{}
\renewcommand{\headrulewidth}{0.4pt}
\renewcommand{\footrulewidth}{0.4pt}


\usepackage{amsmath}

\usepackage{xeCJK}
\setCJKmainfont{SimSun}

\bibliographystyle{IEEEtran}

\newcommand{\half}{\frac{1}{2}}	% 1/2
\renewcommand{\figurename}{图}
\renewcommand{\tablename}{表}

\title{无透镜傅里叶全息显微}
\author{左元 SA13006060}
\date{\today}


\begin{document}

\maketitle

\section{摘要}
数字全息技术是光学成像领域一项新兴技术,
而数字全息显微技术则是应用这种方法实现显微成像的。
在本文中,我首先回顾一下数字全息显微技术的背景,
接着推导一下成像的理论,并根据理论进行数值仿真,
验证理论的正确性。

\section{简介}
为了提高电子显微镜的分辨能力,Dennis Gabor发明于1948年被全息术。
他意识到电子束的衍射图样包含了物象的所有信息,
因此,通过记录衍射场,可以重建目标的物场。
因为这种方法可以记录整个光场,所以他称之为全息术(holography)\cite{gabor1948new, kim2010principles}。

全息术马上被应用到可见光成像领域,
但是直到两项关键技术的出现,它的潜能才被完全挖掘出来。
这两项关键技术分别是激光器为代表的理想的相干光源,
还有就是Leith和Upatnieks发明的分立参考光离轴照明技术\cite{leith1962reconstructed}。
这种照明方式解决了Gabor装置中,重建过程中实像、0级干涉像和共轭像的分离。
之后,全息术的应用得到飞速发展,现在已经是一个成熟的领域了。

对于很多领域的应用来说,实现实施的处理是非常有用的,
但是用传统实现全息的方式却非常难。
数字全息术通过利用电子设备代替物理和化学的记录方式,
利用数值计算模拟光学重建过程,很好的解决了这一问题。
光场传播可以通过衍射理论进行精确地描述,
在1967年,Goodman和Lawrence从摄像机中得到的傅里叶全息图中,
通过数值重建的方法证实了数值重建的可行性\cite{goodman1967digital}。
而在1994年,德国科学家Schnars和Jueptner采用离轴菲涅尔全息记录光路,
用CCD记录了一个骰子的全息图,并重建出了清晰的物场图像\cite{schnars1994direct}。

数字全息显微是通过数字全息术实现显微成像的技术。
它被广泛应用到生命科学、医学等领域,
国际上数字全息显微成像的分辨率已经达到横向亚微米量级、轴向纳米量级\cite{kemper2008digital,marquet2005digital,mann2006quantitative}。
近年来,数字全息显微成像开始考虑向廉价的实现发展,
通过无透镜傅里叶全息光路,可以在手机上通过增加少量的外设实现\cite{breslauer2009mobile,tseng2010lensfree,vashist2014cellphone}。
采用这种方式,将有可能让数字全息显微的技术走向每个家庭。

\section{数字全息成像的理论分析}


\section{无透镜傅里叶全息显微}

\section{总结}


\bibliographystyle{IEEEtran}
\bibliography{ref}

\end{document}